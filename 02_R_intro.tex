\section{Introduction to R}


R is a statistical programming language that is one of the two most used tools for Data Scientists.  Before going into the details of R, I would like to go into the thinkings behind R.  Everything is designed with a purpose, and understanding the context will help you make sense of it in the long run.  

Therefore this chapter will include some introductory notes on the background of R and also the concepts used in R.  We will also go through some practical examples in using R as a programming language.  

\subsection{Story of R so far}

Roots in S 

R is C++ based 

R is a high level programming language 

Alternative to R is python in Data Science.  Each has its strengths and weaknesses.  You should keep an open mind 

The power of R lies in three things:

ease of use 
open source 
powerful packages


All these things are connected with one another, for very good reason and all these leads to one powerful trait: Community of users.  If we take a step back, you would realize that this is true for all successful programming languages.  Next time when you are learning a new programming language, remember to ask yourself these two questions:

\begin{itemize}
\item Where do people ask questions?
\item How quickly are the questions answered?
\end{itemize}

\subsection{Downloading and Installing R}

CRAN is the place where R project is maintained.  This is related to the community.  

\subsection{Basic Operations in R}

Addition, subtraction, multiplying and division.  

The concept of Data Frame and Data Table.  

Load a library for correlation and plot pretty graphs.  

\subsection{Theories of R}

Programming language is like a spoken language in the sense that it has its own spirit and semantics.  With semantics, it comes with style, concepts and even theories of writing good R code.  

Although this is a bit advanced for you, but please remember the following:

\begin{itemize}
\item Concept of tidy table, because R is optimized on row and matrix operations 
\item Follow the google R Style Code when in doubt
\item Always comment 
\end{itemize}

\newpage