\section{Introduction}

This is an internal training programme in Monty Python Group.  Many companies would normally outsource most of the technical tasks externally, while maintaining only a very small team of non-technical managers to oversee the technical gatekeeping.  While we refrain from commenting on business practices, we believe that this is not the DNA of Monty Python.  

\subsection{Why this course?}

The spirit of Monty Python is to have a technical internal team with reasonable capabilities.  This is maintained for several reasons:

\begin{itemize}
\item Having a technical team allows us to have the capacity to have greater ownership of the technical assets of the group, which ultimately brings greater returns to the group
\item Having a relatively strong technical team allows us to know the limitations and boundaries of the solution space.  This is hard to measure but ultimately 
\item There is a natural curiosity that us as humans maintain.  In the age of Decision Science, which every decision we make can allegedly be influenced by an algorithm, it is only natural for us to wonder how, why and the value behind such technical advancement.  This point is perhaps the least business centric, but the most important in my opinion.  
\end{itemize}

The purpose of this course is not to train a Data Scientist, which we leave that to the second part of the Big Data University.  The purpose of this course is to provide an overview of common Data Science techniques in the realms of Customer Relations Management.  

Dull it may seems, as the eye-candy of Data Science in this era are all silver bullets, life-saving medical algorithms, human-beating Go programs and driver-less cars.  Nevertheless, we should be marveling whatever is pushing from behind, rather than the glamour on the surface.  The one thing that connects the success story of image recognition and life-saving algorithms is Machine Learning, which can be manifested through examples in CRM.  

\subsection{Learning Objectives}

This is nevertheless a short course with high expectations.  We hope that everyone who participate in this course will have the effort to go through each example and understand the marvels of human advancement in achieving intelligence from man-made objects.  

But more practically, we hope that by going through this course, participants will be able to impair these techniques in a retail environment.  

There are in total 9 lessons, and each lesson is use-case oriented.  R is used extensively through lessons 3-8, and it is advised that the user can practice using R in their free time where possible.  

\newpage