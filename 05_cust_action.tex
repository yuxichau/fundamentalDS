\section{Predicting Customer Action}

``Action'' here means a definitive set of behaviors that a customer will undergo.  This can be churn, where the customer completely lapses and does not engage again.  Or a customer decides to reengage, spend more, spend on a new item, etc.  These problems are fundamentally different from Sec. \ref{sec:05_cust_value}, where we look a value, which is a scale.  

Throughout this section, we will explore the common techniques of using classification to determine a customer's likelihood of doing an action.  

\subsection{Defining an Action}

In the general sense, there is only one way of defining a customer action, which is when a customer satisfies a criteria that you have set.  

For example, an action might be a customer deciding to purchase a new type of product hierarchy.  

Go through an example.  

Deciding on the condition would mean that it is 

\subsection{Build Features}

``Features'' here mean things that are related to the action.  Focus on the features that you think might be related.  These are where your domain knowledge comes in.  

Feature engineering can be extremely influential to the end results.  This is often the place that is being overlooked.  In fact, top data scientists would argue that 90\% of the success of the machine learning problem is dependent on the quality and mix of features.  

\subsection{Using Logistic Regression}

One of the most basic versions of classification is logistic regression.  



\newpage 